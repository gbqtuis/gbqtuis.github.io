\documentclass[a4paper,12pt]{article}
\usepackage{amssymb}
\usepackage{amsmath}
\usepackage{graphicx}
\usepackage{epsfig}
\usepackage{mathtools}
\usepackage{fancyhdr}
\usepackage{multicol,caption}
\usepackage{color}
\usepackage{cite}
\usepackage{chemfig,chemmacros}
\chemsetup{modules=all}
\usepackage[version=4]{mhchem}
\usepackage{hyperref}
\hypersetup{colorlinks=true,linkcolor=blue,filecolor=blue,urlcolor=blue}
\pagestyle{fancy}
\fancyhf{}
\renewcommand{\headrulewidth}{2pt}
\headheight=30pt
\chead{\large{\bf SEMINARIO DE BIOQUÍMICA TEÓRICA }}
\cfoot{}
\renewcommand{\refname}{}

\begin{document}
\begin{center}
	{\bf Discrepancia teoría-experimento en los niveles de energía del helio} \\
	Prof. Jhon Fredy Pérez Torres \\
	14.07.2025
\end{center}

	En este seminario se presentará el desarrollo histórico del 
	descubrimeinto de los niveles de energ\'ia del átomo de hidrógeno,
	tanto de fuentes experimentales como de cálculos teóricos
	\cite{Duarte2024,Bezginov2019,Scheidegger2024},
	sirviendo de introducción a sistemas de dos electrones (problema de tres cuerpos).
	Luego se discutirá la discrepancia actual que existe entre medidas
	experimentales y cálculos teóricos en el helio, incluyendo su isótopo
	³He y sus correspondientes sistemas muónicos \cite{Clausen2025,Muli2025}.
	La discrepancia m\'as fuerte ocurre en el estado $\rm {^3S_1}(1s^12s^1)$ \cite{Clausen2025}.
	Finalmente, se comentar\'a sobre la ruptura de la simetr\'ia entre los is\'omeros
	\'opticos $\Lambda$ y $\Delta$ en los sistemas moleculares \ce{Ru(acac)3}
	y \ce{Os(acac)3} \cite{RuAcac3}, y su relaci\'on con el problema del \'atomo de helio.

\vspace{1cm}\hrule
\begin{thebibliography}{9}
	\footnotesize
	\bibitem{Duarte2024} Análisis histórico del origen de la ecuación de Dirac,
		Universidad Pedagógica Nacional, Bogotá (2024)
	\bibitem{Bezginov2019} \href{https://doi.org/10.1126/science.aau7807}{Science 365, 1007 (2019)}
	\bibitem{Scheidegger2024} \href{https://doi.org/10.1103/PhysRevLett.132.113001}{Phys. Rev. Lett. 132, 113001 (2024)}
	\bibitem{Clausen2025} \href{https://doi.org/10.1103/PhysRevA.111.012817}{Phys. Rev. Lett. 111, 012817 (2025)}
	\bibitem{Muli2025} \href{https://doi.org/10.1103/PhysRevLett.110.032502}{Phys. Rev. Lett. 032502 (2025)}
	\bibitem{RuAcac3} \href{https://doi.org/10.1021/acs.jpclett.2c02434}{J. Chem. Phys. Lett. 13, 10011 (2022)}
\end{thebibliography}


\end{document}
